% Options for packages loaded elsewhere
\PassOptionsToPackage{unicode}{hyperref}
\PassOptionsToPackage{hyphens}{url}
%
\documentclass[
  man,floatsintext]{apa6}
\usepackage{amsmath,amssymb}
\usepackage{iftex}
\ifPDFTeX
  \usepackage[T1]{fontenc}
  \usepackage[utf8]{inputenc}
  \usepackage{textcomp} % provide euro and other symbols
\else % if luatex or xetex
  \usepackage{unicode-math} % this also loads fontspec
  \defaultfontfeatures{Scale=MatchLowercase}
  \defaultfontfeatures[\rmfamily]{Ligatures=TeX,Scale=1}
\fi
\usepackage{lmodern}
\ifPDFTeX\else
  % xetex/luatex font selection
\fi
% Use upquote if available, for straight quotes in verbatim environments
\IfFileExists{upquote.sty}{\usepackage{upquote}}{}
\IfFileExists{microtype.sty}{% use microtype if available
  \usepackage[]{microtype}
  \UseMicrotypeSet[protrusion]{basicmath} % disable protrusion for tt fonts
}{}
\makeatletter
\@ifundefined{KOMAClassName}{% if non-KOMA class
  \IfFileExists{parskip.sty}{%
    \usepackage{parskip}
  }{% else
    \setlength{\parindent}{0pt}
    \setlength{\parskip}{6pt plus 2pt minus 1pt}}
}{% if KOMA class
  \KOMAoptions{parskip=half}}
\makeatother
\usepackage{xcolor}
\usepackage{longtable,booktabs,array}
\usepackage{calc} % for calculating minipage widths
% Correct order of tables after \paragraph or \subparagraph
\usepackage{etoolbox}
\makeatletter
\patchcmd\longtable{\par}{\if@noskipsec\mbox{}\fi\par}{}{}
\makeatother
% Allow footnotes in longtable head/foot
\IfFileExists{footnotehyper.sty}{\usepackage{footnotehyper}}{\usepackage{footnote}}
\makesavenoteenv{longtable}
\usepackage{graphicx}
\makeatletter
\newsavebox\pandoc@box
\newcommand*\pandocbounded[1]{% scales image to fit in text height/width
  \sbox\pandoc@box{#1}%
  \Gscale@div\@tempa{\textheight}{\dimexpr\ht\pandoc@box+\dp\pandoc@box\relax}%
  \Gscale@div\@tempb{\linewidth}{\wd\pandoc@box}%
  \ifdim\@tempb\p@<\@tempa\p@\let\@tempa\@tempb\fi% select the smaller of both
  \ifdim\@tempa\p@<\p@\scalebox{\@tempa}{\usebox\pandoc@box}%
  \else\usebox{\pandoc@box}%
  \fi%
}
% Set default figure placement to htbp
\def\fps@figure{htbp}
\makeatother
\setlength{\emergencystretch}{3em} % prevent overfull lines
\providecommand{\tightlist}{%
  \setlength{\itemsep}{0pt}\setlength{\parskip}{0pt}}
\setcounter{secnumdepth}{-\maxdimen} % remove section numbering
% Make \paragraph and \subparagraph free-standing
\makeatletter
\ifx\paragraph\undefined\else
  \let\oldparagraph\paragraph
  \renewcommand{\paragraph}{
    \@ifstar
      \xxxParagraphStar
      \xxxParagraphNoStar
  }
  \newcommand{\xxxParagraphStar}[1]{\oldparagraph*{#1}\mbox{}}
  \newcommand{\xxxParagraphNoStar}[1]{\oldparagraph{#1}\mbox{}}
\fi
\ifx\subparagraph\undefined\else
  \let\oldsubparagraph\subparagraph
  \renewcommand{\subparagraph}{
    \@ifstar
      \xxxSubParagraphStar
      \xxxSubParagraphNoStar
  }
  \newcommand{\xxxSubParagraphStar}[1]{\oldsubparagraph*{#1}\mbox{}}
  \newcommand{\xxxSubParagraphNoStar}[1]{\oldsubparagraph{#1}\mbox{}}
\fi
\makeatother
% definitions for citeproc citations
\NewDocumentCommand\citeproctext{}{}
\NewDocumentCommand\citeproc{mm}{%
  \begingroup\def\citeproctext{#2}\cite{#1}\endgroup}
\makeatletter
 % allow citations to break across lines
 \let\@cite@ofmt\@firstofone
 % avoid brackets around text for \cite:
 \def\@biblabel#1{}
 \def\@cite#1#2{{#1\if@tempswa , #2\fi}}
\makeatother
\newlength{\cslhangindent}
\setlength{\cslhangindent}{1.5em}
\newlength{\csllabelwidth}
\setlength{\csllabelwidth}{3em}
\newenvironment{CSLReferences}[2] % #1 hanging-indent, #2 entry-spacing
 {\begin{list}{}{%
  \setlength{\itemindent}{0pt}
  \setlength{\leftmargin}{0pt}
  \setlength{\parsep}{0pt}
  % turn on hanging indent if param 1 is 1
  \ifodd #1
   \setlength{\leftmargin}{\cslhangindent}
   \setlength{\itemindent}{-1\cslhangindent}
  \fi
  % set entry spacing
  \setlength{\itemsep}{#2\baselineskip}}}
 {\end{list}}
\usepackage{calc}
\newcommand{\CSLBlock}[1]{\hfill\break\parbox[t]{\linewidth}{\strut\ignorespaces#1\strut}}
\newcommand{\CSLLeftMargin}[1]{\parbox[t]{\csllabelwidth}{\strut#1\strut}}
\newcommand{\CSLRightInline}[1]{\parbox[t]{\linewidth - \csllabelwidth}{\strut#1\strut}}
\newcommand{\CSLIndent}[1]{\hspace{\cslhangindent}#1}
\ifLuaTeX
\usepackage[bidi=basic]{babel}
\else
\usepackage[bidi=default]{babel}
\fi
\babelprovide[main,import]{english}
% get rid of language-specific shorthands (see #6817):
\let\LanguageShortHands\languageshorthands
\def\languageshorthands#1{}
\ifLuaTeX
  \usepackage[english]{selnolig} % disable illegal ligatures
\fi
% Manuscript styling
\usepackage{upgreek}
\captionsetup{font=singlespacing,justification=justified}

% Table formatting
\usepackage{longtable}
\usepackage{lscape}
% \usepackage[counterclockwise]{rotating}   % Landscape page setup for large tables
\usepackage{multirow}		% Table styling
\usepackage{tabularx}		% Control Column width
\usepackage[flushleft]{threeparttable}	% Allows for three part tables with a specified notes section
\usepackage{threeparttablex}            % Lets threeparttable work with longtable

% Create new environments so endfloat can handle them
% \newenvironment{ltable}
%   {\begin{landscape}\centering\begin{threeparttable}}
%   {\end{threeparttable}\end{landscape}}
\newenvironment{lltable}{\begin{landscape}\centering\begin{ThreePartTable}}{\end{ThreePartTable}\end{landscape}}

% Enables adjusting longtable caption width to table width
% Solution found at http://golatex.de/longtable-mit-caption-so-breit-wie-die-tabelle-t15767.html
\makeatletter
\newcommand\LastLTentrywidth{1em}
\newlength\longtablewidth
\setlength{\longtablewidth}{1in}
\newcommand{\getlongtablewidth}{\begingroup \ifcsname LT@\roman{LT@tables}\endcsname \global\longtablewidth=0pt \renewcommand{\LT@entry}[2]{\global\advance\longtablewidth by ##2\relax\gdef\LastLTentrywidth{##2}}\@nameuse{LT@\roman{LT@tables}} \fi \endgroup}

% \setlength{\parindent}{0.5in}
% \setlength{\parskip}{0pt plus 0pt minus 0pt}

% Overwrite redefinition of paragraph and subparagraph by the default LaTeX template
% See https://github.com/crsh/papaja/issues/292
\makeatletter
\renewcommand{\paragraph}{\@startsection{paragraph}{4}{\parindent}%
  {0\baselineskip \@plus 0.2ex \@minus 0.2ex}%
  {-1em}%
  {\normalfont\normalsize\bfseries\itshape\typesectitle}}

\renewcommand{\subparagraph}[1]{\@startsection{subparagraph}{5}{1em}%
  {0\baselineskip \@plus 0.2ex \@minus 0.2ex}%
  {-\z@\relax}%
  {\normalfont\normalsize\itshape\hspace{\parindent}{#1}\textit{\addperi}}{\relax}}
\makeatother

\makeatletter
\usepackage{etoolbox}
\patchcmd{\maketitle}
  {\section{\normalfont\normalsize\abstractname}}
  {\section*{\normalfont\normalsize\abstractname}}
  {}{\typeout{Failed to patch abstract.}}
\patchcmd{\maketitle}
  {\section{\protect\normalfont{\@title}}}
  {\section*{\protect\normalfont{\@title}}}
  {}{\typeout{Failed to patch title.}}
\makeatother

\usepackage{xpatch}
\makeatletter
\xapptocmd\appendix
  {\xapptocmd\section
    {\addcontentsline{toc}{section}{\appendixname\ifoneappendix\else~\theappendix\fi: #1}}
    {}{\InnerPatchFailed}%
  }
{}{\PatchFailed}
\makeatother
\usepackage{csquotes}
\usepackage{bookmark}
\IfFileExists{xurl.sty}{\usepackage{xurl}}{} % add URL line breaks if available
\urlstyle{same}
\hypersetup{
  pdftitle={Data Science Exam},
  pdfauthor={Precious Nhamo},
  pdflang={en-EN},
  hidelinks,
  pdfcreator={LaTeX via pandoc}}

\title{Data Science Exam}
\author{Precious Nhamo\textsuperscript{}}
\date{}


\shorttitle{SHORT TITLE}

\affiliation{\phantom{0}}

\abstract{%
This document comprises the responses to \textbf{Questions 1 to 5} of the 2025 Data Science examination,
along with separate analyses (e.g., a PowerPoint presentation) completed in accordance with the exam
instructions and organised within the designated folder.
}



\begin{document}
\maketitle

\section{Question 1 Baby Names}\label{question-1-baby-names}

\begin{figure}
\includegraphics[width=0.9\linewidth]{../Question1/Results/nametime} \caption{ }\label{fig:bubble-image-1}
\end{figure}
\begin{figure}
\includegraphics[width=0.9\linewidth]{../Question1/Results/yearonyear} \caption{ }\label{fig:bubble-image-2}
\end{figure}
\begin{figure}
\includegraphics[width=0.9\linewidth]{../Question1/Results/namepopularity} \caption{ }\label{fig:bubble-image-3}
\end{figure}
\begin{figure}
\includegraphics[width=0.9\linewidth]{../Question1/Results/fadsonewonders} \caption{ }\label{fig:bubble-image-4}
\end{figure}
\begin{figure}
\includegraphics[width=0.9\linewidth]{../Question1/Results/mediaimpact} \caption{ }\label{fig:bubble-image-5}
\end{figure}
\begin{figure}
\includegraphics[width=0.9\linewidth]{../Question1/Results/musicimpact} \caption{ }\label{fig:bubble-image-6}
\end{figure}
\begin{figure}
\includegraphics[width=0.9\linewidth]{../Question1/Results/bubble} \caption{ }\label{fig:bubble-image-7}
\end{figure}

\subsubsection{\texorpdfstring{\textbf{Baby Naming Trends in the U.S.}}{Baby Naming Trends in the U.S.}}\label{baby-naming-trends-in-the-u.s.}

\paragraph{\texorpdfstring{\textbf{Prepared for: New York-Based Kids' Toy Design Agency}}{Prepared for: New York-Based Kids' Toy Design Agency}}\label{prepared-for-new-york-based-kids-toy-design-agency}

\paragraph{\texorpdfstring{\textbf{Date: June 18, 2025}}{Date: June 18, 2025}}\label{date-june-18-2025}

\paragraph{\texorpdfstring{\textbf{Objective}}{Objective}}\label{objective}

This report provides a strategic analysis of U.S. baby naming trends to inform character name selection for toy design. The analysis focuses on five key graphs, highlighting naming longevity, surges in popularity, gender-based stability, cultural influences, and the rise of disposable names.

\subsubsection{\texorpdfstring{\textbf{Key Findings \& Strategic Insights}}{Key Findings \& Strategic Insights}}\label{key-findings-strategic-insights}

\paragraph{\texorpdfstring{\textbf{1. Longevity of Baby Names in the Top 25}}{1. Longevity of Baby Names in the Top 25}}\label{longevity-of-baby-names-in-the-top-25}

\begin{itemize}
\tightlist
\item
  \textbf{Observation:} Certain names exhibit remarkable persistence in popularity, remaining in the top 25 for over a decade.
\item
  \textbf{Strategic Insight:} Names with historical longevity (e.g., William, Elizabeth, James, Mary) signal cultural stability and trustworthiness, making them ideal for classic toy characters.
\end{itemize}

\paragraph{\texorpdfstring{\textbf{2. Year-on-Year Surges in Popularity}}{2. Year-on-Year Surges in Popularity}}\label{year-on-year-surges-in-popularity}

\begin{itemize}
\tightlist
\item
  \textbf{Observation:} Names experience sudden spikes due to external influences such as pop culture, politics, and media.
\item
  \textbf{Example:} The name ``Linda'' surged by \textbf{89.1\% in 1947}, likely influenced by Hit song ``Linda'' (1946)by Jack Lawrence.
\item
  \textbf{Strategic Insight:} Leveraging trending names from movies, music, and social movements can enhance toy market relevance.
\end{itemize}

\paragraph{\texorpdfstring{\textbf{3. Gender-Based Stability in Naming Trends}}{3. Gender-Based Stability in Naming Trends}}\label{gender-based-stability-in-naming-trends}

\begin{itemize}
\tightlist
\item
  \textbf{Observation:} Boys' names exhibit \textbf{15-20\% greater stability} in popularity compared to girls' names.
\item
  \textbf{Strategic Insight:} Male character names emphasize tradition and reliability, while female names can reflect evolving cultural trends.Or parents' likely conservatism with boys' names compared to girls' which are trend-sensitive
\end{itemize}

\paragraph{\texorpdfstring{\textbf{4. Influence of HBO \& Billboard Artists on Naming Trends}}{4. Influence of HBO \& Billboard Artists on Naming Trends}}\label{influence-of-hbo-billboard-artists-on-naming-trends}

\begin{itemize}
\tightlist
\item
  \textbf{Observation:} TV shows and music significantly impact baby name choices.
\item
  \textbf{Example:} Names like ``Whitney'' surged in the 1980s due to Whitney Houston's popularity.
\item
  \textbf{Strategic Insight:} Incorporating names inspired by entertainment icons can enhance brand appeal.Integrating a real-time monitoring of children's media into product-development.
\end{itemize}

\paragraph{\texorpdfstring{\textbf{5. The Rise of Disposable Baby Names}}{5. The Rise of Disposable Baby Names}}\label{the-rise-of-disposable-baby-names}

\begin{itemize}
\tightlist
\item
  \textbf{Observation:} One-time top 25 names became \textbf{5x more common after 1960}, indicating a shift toward short-lived naming trends showing decadal patterns (confirming the suspicion).
\item
  \textbf{Strategic Insight:} Short-lived trendy names may work well for limited-edition toy lines or shorter product cycles and name refresh strategies , but should must be balanced with timeless names for long-term brand equity.
\end{itemize}

\subsubsection{\texorpdfstring{\textbf{Recommendations}}{Recommendations}}\label{recommendations}

\begin{enumerate}
\def\labelenumi{\arabic{enumi}.}
\tightlist
\item
  \textbf{Blend Classic \& Trendy Names:} Use a mix of historically persistent names and trending names to maximize appeal.Maybe , legacy lines using persistent names (\textgreater{} 100 ), trendy lines using current events and decade lines that envoke nostalgic names from specific era like Whitney.
\item
  \textbf{Leverage Pop Culture Data:} Monitor entertainment trends to predict future naming spikes.
\item
  \textbf{Gender-Based Strategy:} Design male characters with stable, traditional names and female characters with dynamic, evolving names.
\item
  \textbf{Optimise for Longevity:} Prioritise names with proven staying power for flagship toy lines.You never go with Elizabeth/William
\item
  \textbf{Monitor Disposable Name Trends:} Use short-lived names for seasonal or limited-edition products.
\end{enumerate}

\begin{center}\rule{0.5\linewidth}{0.5pt}\end{center}

\subsubsection{\texorpdfstring{\textbf{Additional Research \& References}}{Additional Research \& References}}\label{additional-research-references}

To further refine the analysis, external research was conducted on baby naming trends, factors influencing names, and naming longevity:

\begin{itemize}
\tightlist
\item
  \textbf{General Trends in U.S. Baby Names:} \href{https://github.com/brettc17/Baby-Name-Trends-Exploring-American-Naming-Patterns}{GitHub Analysis}
\item
  \textbf{Political \& Cultural Influences on Naming:} \href{https://dailytrojan.com/2025/06/18/parents-political-beliefs-shape-us-baby-names/}{Daily Trojan}
\item
  \textbf{Psychological \& Sociological Factors in Naming:} \href{https://lifebyname.com/the-science/}{Life by Name}
\item
  \textbf{Historical Longevity of Baby Names:} \href{https://www.americannamesociety.org/long-term-sociolinguistics-trends-and-phonological-patterns-of-american-names/}{American Name Society}
\end{itemize}

\section{Question 2 Music Taste}\label{question-2-music-taste}

\begin{figure}

{\centering \includegraphics[width=0.9\linewidth]{../Question2/Results/popalbums} 

}

\caption{ }\label{fig:include-image-1}
\end{figure}
\begin{figure}

{\centering \includegraphics[width=0.9\linewidth]{../Question2/Results/uniquesongs} 

}

\caption{ }\label{fig:include-image-2}
\end{figure}
\begin{figure}

{\centering \includegraphics[width=0.9\linewidth]{../Question2/Results/audioeffects} 

}

\caption{ }\label{fig:include-image-3}
\end{figure}
\begin{figure}

{\centering \includegraphics[width=0.9\linewidth]{../Question2/Results/extraeffects} 

}

\caption{ }\label{fig:include-image-4}
\end{figure}
\begin{figure}

{\centering \includegraphics[width=0.9\linewidth]{../Question2/Results/directcompetition} 

}

\caption{ }\label{fig:include-image-5}
\end{figure}
\begin{figure}

{\centering \includegraphics[width=0.9\linewidth]{../Question2/Results/industrytrend} 

}

\caption{ }\label{fig:include-image-6}
\end{figure}

\subsection{Longevity and Musical Progression of Coldplay and Metallica}\label{longevity-and-musical-progression-of-coldplay-and-metallica}

The data reveals distinct trajectories for Coldplay and Metallica in terms of popularity, musical evolution, and industry adaptation. Coldplay demonstrated early dominance, charting five songs in their first decade compared to Metallica's one (Figure 2). Their popularity scores on Spotify also show broader appeal, with a higher median and narrower interquartile range than Metallica's (Figure 1).

Musically, Coldplay's tempo has remained stable (Figure 3), while their audio features, such as danceability and valence, trended positively over time (Figure 4). Metallica, conversely, maintained higher instrumentalness and energy, reflecting their heavier style.

Billboard data highlights their adaptation to industry shifts: Metallica peaked during the album era (1991), while Coldplay thrived post-2000, leveraging streaming's rise (Figure 6). During direct competition (1996+), Coldplay's momentum outpaced Metallica's (Figure 5).

Coldplay's consistent, accessible sound contrasts with Metallica's enduring heavy metal identity. Both bands exemplify longevity but reflect divergent strategies in navigating musical trends

\section{Question 3 : Netflix Content Strategy Analysis}\label{question-3-netflix-content-strategy-analysis}

In light of Netflix's recent subscriber attrition and share price volatility, a strategic review was conducted to inform potential market entry for a new streaming venture. This review draws on IMDb ratings and global production data to assess what drives success in streaming content.

\begin{figure}

{\centering \includegraphics[width=0.9\linewidth]{../Question3/Results/audienceengage} 

}

\caption{ }\label{fig:audienceengage-1}
\end{figure}
\begin{figure}

{\centering \includegraphics[width=0.9\linewidth]{../Question3/Results/countryrating} 

}

\caption{ }\label{fig:audienceengage-2}
\end{figure}
\begin{figure}

{\centering \includegraphics[width=0.9\linewidth]{../Question3/Results/emergers} 

}

\caption{ }\label{fig:audienceengage-3}
\end{figure}
\begin{figure}

{\centering \includegraphics[width=0.9\linewidth]{../Question3/Results/genre_rating_plot} 

}

\caption{ }\label{fig:audienceengage-4}
\end{figure}
\begin{figure}

{\centering \includegraphics[width=0.9\linewidth]{../Question3/Results/genrescore} 

}

\caption{ }\label{fig:audienceengage-5}
\end{figure}
\begin{figure}

{\centering \includegraphics[width=0.9\linewidth]{../Question3/Results/IMDbratings} 

}

\caption{ }\label{fig:audienceengage-6}
\end{figure}
\begin{figure}

{\centering \includegraphics[width=0.9\linewidth]{../Question3/Results/top5movies} 

}

\caption{ }\label{fig:audienceengage-7}
\end{figure}
\begin{figure}

{\centering \includegraphics[width=0.9\linewidth]{../Question3/Results/viewership} 

}

\caption{ }\label{fig:audienceengage-8}
\end{figure}
\begin{figure}

{\centering \includegraphics[width=0.9\linewidth]{../Question3/Results/top20creators} 

}

\caption{ }\label{fig:audienceengage-9}
\end{figure}

\subsection{Key Findings}\label{key-findings}

\subsubsection{1. Quality vs.~Volume Trade-Off}\label{quality-vs.-volume-trade-off}

Japanese titles, while constituting only 1.6\% of Netflix's catalogue, achieve superior average IMDb ratings (mean = 6.7, SD = ±0.3). By contrast, the United States and India collectively contribute over 50\% of Netflix's content but yield lower average ratings (mean = 6.1--6.3) (Figures 8, 15)
\textbf{Implication}: High-volume strategies may dilute perceived quality.

\subsubsection{2. Genre Performance Differentials}\label{genre-performance-differentials}

Content genres vary markedly in audience reception. Documentaries (mean = 7.6) and dramas (mean = 7.0) lead on quality metrics, while action and comedy lag behind (mean = 6.2 and 6.5 respectively) (Figure 11). Crime and thriller categories also demonstrate strong viewer engagement.

\textbf{Implication}:Narrative depth and authenticity drive audience satisfaction.

\subsubsection{3. Optimal Runtime Windows}\label{optimal-runtime-windows}

Ratings peak for films with runtimes between 90--120 minutes (mean = 7.5). Shorter or longer formats tend to correlate with lower ratings (Figure 12).
\textbf{Implication}:Viewer engagement is maximised within a standardised runtime threshold.

\subsubsection{4. Age and Longevity of Content}\label{age-and-longevity-of-content}

Older films (\textgreater20 years) sustain higher average ratings (mean = 8.5)(Figure 7) compared to recent productions (1--3 years; mean = 6.0)(Figure 14).
\textbf{Implication}:Legacy content benefits from curation and nostalgia; newer content faces saturation and discovery challenges.

\subsection{Strategic Recommendations for New Market Entrant}\label{strategic-recommendations-for-new-market-entrant}

To position competitively against incumbent platforms:

\subsubsection{1. Targeted Content Acquisition}\label{targeted-content-acquisition}

Prioritise \textbf{high-quality, underrepresented niches}---such as Japanese documentaries and international dramas---to differentiate on quality and build a prestige brand image.

\subsubsection{2. Genre Investment Strategy}\label{genre-investment-strategy}

Allocate production and licensing budgets towards \textbf{high-performing genres} (e.g., crime, thriller, drama). Avoid overserved categories like generic comedy unless uniquely positioned.

\subsubsection{3. Runtime Standardisation}\label{runtime-standardisation}

Anchor commissioned or licensed content within the 90--120 minute range to enhance user engagement and completion rates.

\subsection{Conclusion}\label{conclusion}

A differentiated, quality-first strategy that leverages genre insights, runtime discipline, and curated international content provides a clear pathway for new entrants to gain market share and investor confidence---without replicating the scale-driven pitfalls observed in Netflix's recent trajectory.

\section{Question 4 Billionaires}\label{question-4-billionaires}

\begin{figure}
\includegraphics[width=0.9\linewidth]{../Question4/Results/wealthmap} \caption{ }\label{fig:wealthmap-image}
\end{figure}

\begin{figure}
\includegraphics[width=0.9\linewidth]{../Question4/Results/inheritance} \caption{ }\label{fig:inheritance-image}
\end{figure}

\begin{figure}
\includegraphics[width=0.9\linewidth]{../Question4/Results/industrydominance} \caption{ }\label{fig:industrydominance-image}
\end{figure}

\begin{figure}
\includegraphics[width=0.9\linewidth]{../Question4/Results/bubble} \caption{ }\label{fig:bubbles}
\end{figure}

\begin{figure}
\includegraphics[width=0.9\linewidth]{../Question4/Results/wealthbyregion} \caption{ }\label{fig:wealthbyregion-image}
\end{figure}

\begin{figure}
\includegraphics[width=0.9\linewidth]{../Question4/Results/topcountries} \caption{ }\label{fig:topcountries-image}
\end{figure}

\begin{figure}
\includegraphics[width=0.9\linewidth]{../Question4/Results/wealthgrowth} \caption{ }\label{fig:wealthgrowth-image}
\end{figure}

\begin{figure}
\includegraphics[width=0.9\linewidth]{../Question4/Results/industrypermarket} \caption{ }\label{fig:industrypermarket-image}
\end{figure}

\begin{figure}
\includegraphics[width=0.9\linewidth]{../Question4/Results/marketgender} \caption{ }\label{fig:marketgender-image}
\end{figure}

\begin{figure}
\includegraphics[width=0.9\linewidth]{../Question4/Results/regional} \caption{ }\label{fig:regional-image}
\end{figure}

\begin{figure}
\includegraphics[width=0.9\linewidth]{../Question4/Results/genderage} \caption{ }\label{fig:genderage-image}
\end{figure}

\begin{figure}
\includegraphics[width=0.9\linewidth]{../Question4/Results/founder_age_by_market_type} \caption{ }\label{fig:founder-age-image}
\end{figure}

\begin{figure}
\includegraphics[width=0.9\linewidth]{../Question4/Results/billmarket} \caption{ }\label{fig:billmarket-image}
\end{figure}

\begin{figure}
\includegraphics[width=0.9\linewidth]{../Question4/Results/ageave} \caption{ }\label{fig:ageave-image}
\end{figure}

\begin{figure}
\includegraphics[width=0.9\linewidth]{../Question4/Results/age_trends_by_market_type} \caption{ }\label{fig:age-trends-market-image}
\end{figure}

\begin{figure}
\includegraphics[width=0.9\linewidth]{../Question4/Results/age_distribution_by_decade} \caption{ }\label{fig:age-distribution-decade-image}
\end{figure}

\subsection{Data analysis}\label{data-analysis}

We used R (Version 4.4.3; R Core Team, 2025) and the R-packages \emph{dplyr} (Version 1.1.4; Wickham, François, Henry, Müller, \& Vaughan, 2023), \emph{fastDummies} (Version 1.7.5; Kaplan, 2025), \emph{forcats} (Version 1.0.0; Wickham, 2023a), \emph{ggplot2} (Version 3.5.2; Wickham, 2016), \emph{ggrepel} (Version 0.9.6; Slowikowski, 2024), \emph{ggthemes} (Version 5.1.0; Arnold, 2024), \emph{igraph} (Version 2.1.4; Csardi \& Nepusz, 2006), \emph{lubridate} (Version 1.9.4; Grolemund \& Wickham, 2011), \emph{pacman} (Version 0.5.1; Rinker \& Kurkiewicz, 2018), \emph{papaja} (Version 0.1.3; Aust \& Barth, 2024), \emph{patchwork} (Version 1.3.0; Pedersen, 2024), \emph{purrr} (Version 1.0.4; Wickham \& Henry, 2025), \emph{readr} (Version 2.1.5; Wickham, Hester, \& Bryan, 2024), \emph{readxl} (Version 1.4.5; Wickham \& Bryan, 2025), \emph{rnaturalearth} (Version 1.0.1; Massicotte \& South, 2023; South, Michael, \& Massicotte, 2024), \emph{rnaturalearthdata} (Version 1.0.0; South et al., 2024), \emph{scales} (Version 1.3.0; Wickham, Pedersen, \& Seidel, 2023), \emph{sf} (Version 1.0.21; Pebesma, 2018), \emph{stringr} (Version 1.5.1; Wickham, 2023b), \emph{tibble} (Version 3.2.1; Müller \& Wickham, 2023), \emph{tidyr} (Version 1.3.1; Wickham, Vaughan, \& Girlich, 2024), \emph{tidyverse} (Version 2.0.0; Wickham et al., 2019), \emph{tinylabels} (Version 0.2.5; Barth, 2025), \emph{viridis} (Garnier et al., 2023; Version 0.6.5; Garnier et al., 2024), and \emph{viridisLite} (Version 0.4.2; Garnier et al., 2023) for all our analyses.
\#\#Results

\#\#Discussion

\section{Question 5 Health}\label{question-5-health}

\subsection{On Air Script}\label{on-air-script}

For the 9-to-5 gamer: 1 hour of extra sleep beats 1 hour on the treadmill. Data shows stress-free living is your real cheat code.

We've been told for years to `exercise more'---but new data reveals the real game-changers: sleep and stress management. Here's the proof: Adults with poor sleep and high stress gained 7--10 pounds more than those with good sleep and low stress---based on regression analysis (p\textless0.01). The worst hit? Early-career professionals (ages 30--39), who saw up to 10 pounds more weight gain when stressed---even if they exercised. Why this matters: Gym memberships won't fix this. The solution? Protect your sleep: Aim for 7--9 hours, and keep a consistent schedule---it's your metabolic lifeline. Tame stress: Short walks, 5-minute breathing exercises, or setting work boundaries can slash health risks. Bottom line? Small changes to sleep and stress beat marathon workouts. Start tonight---your body will thank you.

\begin{figure}
\includegraphics[width=0.9\linewidth]{../Question5/Results/Sleep_Stress_Boxplot} \caption{ }\label{fig:sleep-stress-boxplot}
\end{figure}

\begin{table}

\caption{(\#tab:sleep-risk-pivot )Weight Change Summary}
\centering
\begin{tabular}[t]{l|l|r|r|r}
\hline
Gender & age\_group & mean\_weight\_change & median\_weight\_change & n\\
\hline
F & Adults (20-29) & -1.0181151 & -0.3680584 & 10\\
\hline
F & Early Career (30-39) & -2.4545587 & -1.3943583 & 11\\
\hline
F & Middle-aged (40-49) & -3.4805916 & 0.7000000 & 12\\
\hline
F & Preretirement (50-64) & -1.5639095 & 0.0000000 & 9\\
\hline
F & Teenagers (<20) & -1.7940640 & -1.7940640 & 1\\
\hline
M & Adults (20-29) & -2.3794779 & -0.1679132 & 14\\
\hline
M & Early Career (30-39) & -5.5847844 & 0.6000000 & 12\\
\hline
M & Middle-aged (40-49) & -3.7051828 & 0.4500000 & 12\\
\hline
M & Preretirement (50-64) & -0.8648233 & -0.3807078 & 14\\
\hline
M & Teenagers (<20) & -5.2528414 & -5.7184960 & 5\\
\hline
\end{tabular}
\end{table}

\begin{table}

\caption{(\#tab:sleep-risk-pivot )Sleep Risk Pivot Summary}
\centering
\begin{tabular}[t]{l|l|r|r}
\hline
Poor Sleep Risk & age\_group & Mean\_Weight\_Change & Count\\
\hline
No & Adults (20-29) & 0.2142665 & 9\\
\hline
No & Early Career (30-39) & 1.5777778 & 9\\
\hline
No & Middle-aged (40-49) & 2.9571429 & 7\\
\hline
No & Preretirement (50-64) & -0.0982816 & 11\\
\hline
No & Teenagers (<20) & 1.9500000 & 2\\
\hline
Yes & Adults (20-29) & -3.0281494 & 15\\
\hline
Yes & Early Career (30-39) & -7.7298256 & 14\\
\hline
Yes & Middle-aged (40-49) & -6.2899584 & 17\\
\hline
Yes & Preretirement (50-64) & -2.0918011 & 12\\
\hline
Yes & Teenagers (<20) & -7.9895678 & 4\\
\hline
\end{tabular}
\end{table}

\begin{table}

\caption{(\#tab:sleep-risk-pivot )Stress Risk Pivot Summary}
\centering
\begin{tabular}[t]{l|l|r|r}
\hline
Poor Sleep Risk & age\_group & Mean\_Weight\_Change & Count\\
\hline
No & Adults (20-29) & 0.2142665 & 9\\
\hline
No & Early Career (30-39) & 1.5777778 & 9\\
\hline
No & Middle-aged (40-49) & 2.9571429 & 7\\
\hline
No & Preretirement (50-64) & -0.0982816 & 11\\
\hline
No & Teenagers (<20) & 1.9500000 & 2\\
\hline
Yes & Adults (20-29) & -3.0281494 & 15\\
\hline
Yes & Early Career (30-39) & -7.7298256 & 14\\
\hline
Yes & Middle-aged (40-49) & -6.2899584 & 17\\
\hline
Yes & Preretirement (50-64) & -2.0918011 & 12\\
\hline
Yes & Teenagers (<20) & -7.9895678 & 4\\
\hline
\end{tabular}
\end{table}

\newpage

\begin{longtable}[]{@{}
  >{\raggedright\arraybackslash}p{(\linewidth - 4\tabcolsep) * \real{0.5138}}
  >{\raggedright\arraybackslash}p{(\linewidth - 4\tabcolsep) * \real{0.2477}}
  >{\raggedright\arraybackslash}p{(\linewidth - 4\tabcolsep) * \real{0.2294}}@{}}
\toprule\noalign{}
\begin{minipage}[b]{\linewidth}\raggedright
\end{minipage} & \begin{minipage}[b]{\linewidth}\raggedright
Sleep-Stress Interaction
\end{minipage} & \begin{minipage}[b]{\linewidth}\raggedright
Stress-Age Interaction
\end{minipage} \\
\midrule\noalign{}
\endhead
\midrule\noalign{}
\multicolumn{3}{@{}>{\raggedright\arraybackslash}p{(\linewidth - 4\tabcolsep) * \real{0.9908} + 4\tabcolsep}@{}}{%
\begin{minipage}[t]{\linewidth}\raggedright
\begin{itemize}
\tightlist
\item
  p \textless{} 0.1, ** p \textless{} 0.05, *** p \textless{} 0.01
\end{itemize}
\end{minipage}} \\
\bottomrule\noalign{}
\endlastfoot
(Intercept) & 1.758 & -0.315 \\
& (6.660) & (1.510) \\
sleep\_riskPoor Sleep & -7.168*** & \\
& (2.704) & \\
stress\_riskLow Stress & 5.800** & \\
& (2.717) & \\
physical\_activityModerately Active & -1.142 & \\
& (1.893) & \\
physical\_activitySedentary & -0.539 & \\
& (1.982) & \\
physical\_activityVery Active & -1.076 & \\
& (2.311) & \\
age\_groupEarly Career (30-39) & -2.976 & 0.653 \\
& (1.875) & (2.135) \\
age\_groupMiddle-aged (40-49) & -2.226 & 0.107 \\
& (1.766) & (2.103) \\
age\_groupPreretirement (50-64) & -0.777 & 0.164 \\
& (1.891) & (2.135) \\
age\_groupTeenagers (\textless20) & -3.178 & -1.867 \\
& (2.861) & (3.094) \\
GenderM & -1.650 & \\
& (1.597) & \\
Daily Caloric Surplus/Deficit & 0.002 & \\
& (0.003) & \\
BMR (Calories) & -0.001 & \\
& (0.002) & \\
Duration (weeks) & -0.158 & \\
& (0.175) & \\
sleep\_riskPoor Sleep × stress\_riskLow Stress & 2.980 & \\
& (3.290) & \\
High Stress RiskYes & & -4.493* \\
& & (2.615) \\
High Stress RiskYes × age\_groupEarly Career (30-39) & & -10.049*** \\
& & (3.785) \\
High Stress RiskYes × age\_groupMiddle-aged (40-49) & & -7.112* \\
& & (3.767) \\
\multicolumn{3}{@{}>{\raggedright\arraybackslash}p{(\linewidth - 4\tabcolsep) * \real{0.9908} + 4\tabcolsep}@{}}{%
High Stress RiskYes × age\_groupPreretirement (50-64) \textbar{} \textbar{} 1.248 \textbar{}} \\
& & (3.785) \\
High Stress RiskYes × age\_groupTeenagers (\textless20) & & -10.476 \\
& & (7.112) \\
Num.Obs. & 100 & 100 \\
R2 & 0.455 & 0.402 \\
R2 Adj. & 0.366 & 0.342 \\
AIC & 655.5 & 654.9 \\
BIC & 697.2 & 683.5 \\
Log.Lik. & -311.752 & -316.433 \\
RMSE & 5.47 & 5.73 \\
\end{longtable}

\subsubsection{\texorpdfstring{\textbf{Key Findings on Health Determinants}}{Key Findings on Health Determinants}}\label{key-findings-on-health-determinants}

\begin{enumerate}
\def\labelenumi{\arabic{enumi}.}
\tightlist
\item
  \textbf{Sleep Quality is Critical}:

  \begin{itemize}
  \tightlist
  \item
    Poor sleep correlates with \textbf{-7.2 lbs} weight gain (\emph{p\textless0.01}), surpassing the impact of physical activity (non-significant coefficients).\\
  \item
    Teens and early-career adults with poor sleep show the worst outcomes (\textbf{-7.7 to -8.0 lbs} mean weight change).Teens are abstracted from main analysis as their n is very small , n =5 .
  \end{itemize}
\item
  \textbf{Stress Drives Negative Outcomes}:

  \begin{itemize}
  \tightlist
  \item
    High stress alone leads to \textbf{-4.5 lbs} weight gain (\emph{p\textless0.1}).\\
  \item
    Stress combined with poor sleep exacerbates effects, especially in younger age groups (interaction terms up to \textbf{-10.0 lbs}, \emph{p\textless0.01}).
  \end{itemize}
\item
  \textbf{Sedentary Lifestyle Modifies Risk}:

  \begin{itemize}
  \tightlist
  \item
    Sedentary individuals (red dots, Figure 16) show higher weight variability, but sleep/stress dominate overall trends.
  \end{itemize}
\end{enumerate}

\subsubsection{\texorpdfstring{\textbf{Practical Recommendations}}{Practical Recommendations}}\label{practical-recommendations}

\begin{itemize}
\tightlist
\item
  \textbf{Prioritise Sleep Hygiene}: Consistent sleep schedules improve metabolic health more than moderate exercise.\\
\item
  \textbf{Stress Management}: Mindfulness or flexible work policies could mitigate high-stress impacts, particularly for younger adults.\\
\item
  \textbf{Targeted Interventions}: Early-career professionals need tailored programs addressing sleep and stress.
\end{itemize}

\newpage

\section{References}\label{references}

\phantomsection\label{refs}
\begin{CSLReferences}{1}{0}
\bibitem[\citeproctext]{ref-R-ggthemes}
Arnold, J. B. (2024). \emph{Ggthemes: Extra themes, scales and geoms for 'ggplot2'}. Retrieved from \url{https://CRAN.R-project.org/package=ggthemes}

\bibitem[\citeproctext]{ref-R-papaja}
Aust, F., \& Barth, M. (2024). \emph{{papaja}: {Prepare} reproducible {APA} journal articles with {R Markdown}}. \url{https://doi.org/10.32614/CRAN.package.papaja}

\bibitem[\citeproctext]{ref-R-tinylabels}
Barth, M. (2025). \emph{{tinylabels}: Lightweight variable labels}. \url{https://doi.org/10.32614/CRAN.package.tinylabels}

\bibitem[\citeproctext]{ref-R-igraph}
Csardi, G., \& Nepusz, T. (2006). The igraph software package for complex network research. \emph{InterJournal}, \emph{Complex Systems}, 1695. Retrieved from \url{https://igraph.org}

\bibitem[\citeproctext]{ref-R-viridisLite}
Garnier, Simon, Ross, Noam, Rudis, Robert, \ldots{} Cédric. (2023). \emph{{viridis(Lite)} - colorblind-friendly color maps for r}. \url{https://doi.org/10.5281/zenodo.4678327}

\bibitem[\citeproctext]{ref-R-viridis}
Garnier, Simon, Ross, Noam, Rudis, Robert, \ldots{} Cédric. (2024). \emph{{viridis(Lite)} - colorblind-friendly color maps for r}. \url{https://doi.org/10.5281/zenodo.4679423}

\bibitem[\citeproctext]{ref-R-lubridate}
Grolemund, G., \& Wickham, H. (2011). Dates and times made easy with {lubridate}. \emph{Journal of Statistical Software}, \emph{40}(3), 1--25. Retrieved from \url{https://www.jstatsoft.org/v40/i03/}

\bibitem[\citeproctext]{ref-R-fastDummies}
Kaplan, J. (2025). \emph{fastDummies: Fast creation of dummy (binary) columns and rows from categorical variables}. Retrieved from \url{https://CRAN.R-project.org/package=fastDummies}

\bibitem[\citeproctext]{ref-R-rnaturalearth}
Massicotte, P., \& South, A. (2023). \emph{Rnaturalearth: World map data from natural earth}. Retrieved from \url{https://CRAN.R-project.org/package=rnaturalearth}

\bibitem[\citeproctext]{ref-R-tibble}
Müller, K., \& Wickham, H. (2023). \emph{Tibble: Simple data frames}. Retrieved from \url{https://CRAN.R-project.org/package=tibble}

\bibitem[\citeproctext]{ref-R-sf}
Pebesma, E. (2018). {Simple Features for R: Standardized Support for Spatial Vector Data}. \emph{{The R Journal}}, \emph{10}(1), 439--446. \url{https://doi.org/10.32614/RJ-2018-009}

\bibitem[\citeproctext]{ref-R-patchwork}
Pedersen, T. L. (2024). \emph{Patchwork: The composer of plots}. Retrieved from \url{https://CRAN.R-project.org/package=patchwork}

\bibitem[\citeproctext]{ref-R-base}
R Core Team. (2025). \emph{R: A language and environment for statistical computing}. Vienna, Austria: R Foundation for Statistical Computing. Retrieved from \url{https://www.R-project.org/}

\bibitem[\citeproctext]{ref-R-pacman}
Rinker, T. W., \& Kurkiewicz, D. (2018). \emph{{pacman}: {P}ackage management for {R}}. Buffalo, New York. Retrieved from \url{http://github.com/trinker/pacman}

\bibitem[\citeproctext]{ref-R-ggrepel}
Slowikowski, K. (2024). \emph{Ggrepel: Automatically position non-overlapping text labels with 'ggplot2'}. Retrieved from \url{https://CRAN.R-project.org/package=ggrepel}

\bibitem[\citeproctext]{ref-R-rnaturalearthdata}
South, A., Michael, S., \& Massicotte, P. (2024). \emph{Rnaturalearthdata: World vector map data from natural earth used in 'rnaturalearth'}. Retrieved from \url{https://CRAN.R-project.org/package=rnaturalearthdata}

\bibitem[\citeproctext]{ref-R-ggplot2}
Wickham, H. (2016). \emph{ggplot2: Elegant graphics for data analysis}. Springer-Verlag New York. Retrieved from \url{https://ggplot2.tidyverse.org}

\bibitem[\citeproctext]{ref-R-forcats}
Wickham, H. (2023a). \emph{Forcats: Tools for working with categorical variables (factors)}. Retrieved from \url{https://CRAN.R-project.org/package=forcats}

\bibitem[\citeproctext]{ref-R-stringr}
Wickham, H. (2023b). \emph{Stringr: Simple, consistent wrappers for common string operations}. Retrieved from \url{https://CRAN.R-project.org/package=stringr}

\bibitem[\citeproctext]{ref-R-tidyverse}
Wickham, H., Averick, M., Bryan, J., Chang, W., McGowan, L. D., François, R., \ldots{} Yutani, H. (2019). Welcome to the {tidyverse}. \emph{Journal of Open Source Software}, \emph{4}(43), 1686. \url{https://doi.org/10.21105/joss.01686}

\bibitem[\citeproctext]{ref-R-readxl}
Wickham, H., \& Bryan, J. (2025). \emph{Readxl: Read excel files}. Retrieved from \url{https://CRAN.R-project.org/package=readxl}

\bibitem[\citeproctext]{ref-R-dplyr}
Wickham, H., François, R., Henry, L., Müller, K., \& Vaughan, D. (2023). \emph{Dplyr: A grammar of data manipulation}. Retrieved from \url{https://CRAN.R-project.org/package=dplyr}

\bibitem[\citeproctext]{ref-R-purrr}
Wickham, H., \& Henry, L. (2025). \emph{Purrr: Functional programming tools}. Retrieved from \url{https://CRAN.R-project.org/package=purrr}

\bibitem[\citeproctext]{ref-R-readr}
Wickham, H., Hester, J., \& Bryan, J. (2024). \emph{Readr: Read rectangular text data}. Retrieved from \url{https://CRAN.R-project.org/package=readr}

\bibitem[\citeproctext]{ref-R-scales}
Wickham, H., Pedersen, T. L., \& Seidel, D. (2023). \emph{Scales: Scale functions for visualization}. Retrieved from \url{https://CRAN.R-project.org/package=scales}

\bibitem[\citeproctext]{ref-R-tidyr}
Wickham, H., Vaughan, D., \& Girlich, M. (2024). \emph{Tidyr: Tidy messy data}. Retrieved from \url{https://CRAN.R-project.org/package=tidyr}

\end{CSLReferences}


\end{document}
